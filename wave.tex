% Adaptado de (BAUER, WESTFALL, DIAS, 2013, p.86, Figura 3.8)
% Adaptado de (BAUER, WESTFALL, 2014, p.466, Figure 15.8)
\documentclass{standalone}
\let\oldvec\vec
\usepackage{amsmath}
\usepackage{tikz}
\usepackage{pgfplots}
\pgfplotsset{compat=1.17}
\usepgfplotslibrary{patchplots}
\usepackage{comment}
\tikzset{>=latex} % for LaTeX arrow head
\usepackage{xcolor}

\pgfplotsset{colormap={wave}{rgb255=(145,85,61) rgb255=(229,200,168) rgb255=(255,240,225) rgb255=(229,200,168) rgb255=(145,85,61)}}

\definecolor{length}{rgb}{0 , 0.5 , 0}
\definecolor{time}{rgb}{0 , 0 , 1}

\usetikzlibrary{arrows}

\begin{document}

\begin{tikzpicture}
\begin{axis}[
    view={-45}{30},
    width=12cm,
    height=6cm,
    xlabel=$x$,
    ylabel=$t$,
    zlabel=$z$,
    zlabel style={rotate=-90},
    zmin=-1,
    zmax=1,
    ticks=none,
    grid=none,
	colormap name=wave
]

% Senoide do comprimento
\addplot3[
	very thick,black,
	domain=0:720,
	domain y=0:1080,
	samples=100,
	samples y=0,
] 
    (x , 1080 , {sin(-x)});

% Senoide do tempo
\addplot3[
	very thick,black,
	domain=0:720,
	domain y=0:1080,
	samples=100,
] 
    (720 , y , {sin(y)});

% Onda
\addplot3[
	surf,shader=interp,
	domain=0:720,
	domain y=0:1080,
	samples=100,
] 
	(x , y , {sin(y - x)});

% Senoide do tempo
\addplot3[
	very thick,time,
	domain=0:720,
	domain y=0:1080,
	samples=100,
] 
    (0 , y , {sin(y)});

% Senoide do comprimento
\addplot3[
	very thick,length,
	domain=0:720,
	domain y=0:1080,
	samples=100,
	samples y=0,
] 
    (x , 0 , {sin(-x)});

% Indicação do período da onda
\draw[time,dashed] ( 0 , 90 , -1 ) -- (0 , 90 , 1);
\draw[time,dashed] ( 0 , 450 , -1 ) -- (0 , 450 , 1);
\draw[time, <->] ( 0 , 90 , 0 ) -- (0 , 450 , 0);
\draw (0 , 270 , 0) node[above]{\color{time}$\mathbf{T}$};

% Indicação do comprimento da onda
\draw[length,dashed] ( 270 , 0 , -1 ) -- ( 270 , 0 , 1);
\draw[length,dashed] ( 630 , 0 , -1 ) -- ( 630 , 0 , 1);
\draw[length, <->] ( 270 , 0 , 0 ) -- (630 , 0 , 0);
\draw (450 , 0 , 0) node[above]{\color{length}$\pmb{\lambda}$};

\end{axis}

\end{tikzpicture}

\end{document}